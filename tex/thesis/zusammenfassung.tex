\vspace*{\fill}
\begin{center}
  \large
    \textbf{Zusammenfassung}\\
  - \textit{Deutsch} -
\end{center}


\begin{flushleft}
\setlength{\leftskip}{1cm}
\setlength{\rightskip}{1cm}
Maschinelles Lernen (\acs{ml}) and künstliche Intelligenz (KI) werden in immer mehr Anwendungsbereichen eingesetzt.
In einigen Umgebungen wirft die Integration dieser Technologien über Dittanbieter- oder Cloud-Dienste jedoch erhebliche Bedenken hinsichtlich des Datenschutzes auf.
Volle homomorphische Verschlüsselung (\acs{fhe}) ermöglicht es \acs{ml}-Modellen mit verschlüsselten Daten zu arbeiten, sodass sensible Nutzungen dieser Modelle ausgelagert werden können, ohne dass private Daten offengelegt werden.

Diese Arbeit untersucht, ob \acs{fhe} dieses Versprechen erfüllt, und ob es derzeit für echte \acs{ml}-Anwendungen geeignet ist.
Zu diesem Zweck haben wir eine Reihe von Experimenten durchgeführt, in denen verschiedene \acs{fhe}-Modelle miteinander, und mit ihren Klartext-Entsprechungen, verglichen wurden.
Wir haben diese Vergleiche für mehrere Datensätze und Klassifizierungsaufgaben durchgeführt, sowohl für verschlüsselte Inferenz als auch Training.
Als einen etwas anspruchsvolleren Anwendungsfall haben wir außerdem zwei maßgeschneiderte Ansätze entwickelt, um die Aufgabe der Erkennung benannter Entitäten (\acs{ner}) homomorph zu lösen.
Darüber hinaus geben wir eine zugängliche Einführung in die theoretischen Grundlagen von \acs{fhe}, um das Verständnis und die Erklärung unserer Resultate zu unterstützen.

Unsere Ergebnisse zeigen, dass \acs{fhe} zwar in eine Reihe von \acs{ml}-Aufgaben wie z.B. die Spam-Erkennung integriert werden kann, jedoch weiterhin erhebliche Limitierungen bestehen.
Dazu gehören ein beträchtlicher Mehraufwand in der Laufzeit sowohl auf Client- als auch auf Serverseite, sowie eine verringerte Genauigkeit bei einigen Modellen.
Angesichts dieser Einschränkungen diskutieren wir auch alternative Ansätze zur Gewährleistung des Datenschutzes im \acs{ml}, wie beispielsweise eine clientseitige Modellausführung.

Das Ziel dieser Arbeit ist es, sowohl \acs{fhe} als Datenschutztechnologie einzuführen, als auch eine informierte Entscheidung darüber zu ermöglichen, ob \acs{fhe} für eine bestimme \acs{ml}-Anwendung geeignet ist, oder ob alternative Techniken möglicherweise eine passendere Lösung darstellen.
\end{flushleft}
\vspace*{\fill}

\acresetall
