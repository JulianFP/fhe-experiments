% This template was initially provided by Dulip Withanage.
% Modifications for the database systems research group
% were made by Conny Junghans,  Jannik Strötgen and Michael Gertz

\documentclass[
     12pt,         % font size
     a4paper,      % paper format
     BCOR=10mm,     % binding correction
     DIV=14,        % stripe size for margin calculation
     liststotoc,   % table listing in toc
     bibtotoc,     % bibliography in toc
%     idxtotoc,     % index in toc
     parskip=half,       % paragraph skip instead of paragraph indent
     openright
     ]{scrreprt}

%%%%%%%%%%%%%%%%%%%%%%%%%%%%%%%%%%%%%%%%%%%%%%%%%%%%%%%%%%%%

% PACKAGES:

% Input and font encoding
\usepackage{lmodern}
\usepackage[utf8]{inputenc}
\usepackage[T1]{fontenc}
% Index-generation
\usepackage{makeidx}
% Einbinden von URLs:
\usepackage{url}
% Special \LaTex symbols (e.g. \BibTeX):
%\usepackage{doc}
% Include Graphic-files:
\usepackage{graphicx}
% Include doc++ generated tex-files:
%\usepackage{docxx}
% Include PDF links
%\usepackage[pdftex, bookmarks=true]{hyperref}

% Fuer anderthalbzeiligen Textsatz
\usepackage{setspace}

% for subfigures (i.e. multiple images side-by-side)
\usepackage{subcaption}

\usepackage{amsmath} %for stuff over arrows
\newcommand{\Mod}[1]{\quad\left(\mathrm{mod}\ #1\right)} %new command for mod's without the leading spaces
\DeclareMathOperator*{\argmax}{arg\,max}
\DeclareMathOperator*{\argmin}{arg\,min}

\usepackage{amssymb}

\usepackage{amsthm}
\theoremstyle{remark}
\newtheorem*{remark}{Remark}

%I switched to BibLatex
\usepackage[backend=biber]{biblatex}
\addbibresource{../references.bib}

% hyperrefs in the documents
\usepackage[bookmarks=true,colorlinks,pdfpagelabels,pdfstartview = FitH,bookmarksopen = true,bookmarksnumbered = true,linkcolor = black,plainpages = false,hypertexnames = false,citecolor = black,urlcolor=black]{hyperref}
%\usepackage{hyperref}

% code listings
\usepackage{listings}
\usepackage{xcolor}
\definecolor{codebackground}{rgb}{1,1,1}
\definecolor{keywordcolor}{rgb}{0,0,0.8}
\definecolor{stringcolor}{rgb}{0.8,0,0.09}
\definecolor{commentcolor}{rgb}{0.07,0.53,0.03}
\definecolor{numbercolor}{rgb}{0.75,0.75,0.75}
\lstdefinestyle{mystyle}{
    backgroundcolor=\color{codebackground},
    commentstyle=\color{commentcolor},
    keywordstyle=\color{keywordcolor},
    morekeywords={as, and},
    numberstyle=\tiny\color{numbercolor},
    stringstyle=\color{stringcolor},
    language=Python,
    basicstyle=\ttfamily\footnotesize,
    breakatwhitespace=false,
    breaklines=true,
    captionpos=b,
    keepspaces=true,
    numbers=left,
    numbersep=5pt,
    showspaces=false,
    showstringspaces=false,
    showtabs=false,
    tabsize=4
}
\lstset{style=mystyle}

%%%%%%%%%%%%%%%%%%%%%%%%%%%%%%%%%%%%%%%%%%%%%%%%%%%%%%%%%%%%

% OTHER SETTINGS:

% Pagestyle:
\pagestyle{headings}

%caption spacing
\setlength{\abovecaptionskip}{3pt}


\begin{document}

% TITLE:
\pagenumbering{roman}
\begin{titlepage}


\vspace*{1cm}
\begin{center}
\vspace*{3cm}
\textbf{
\Large Heidelberg University\\
\smallskip
\Large Faculty of Mathematics and Computer Science\\
\smallskip
\Large Interdisciplinary Center for Scientific Computing\\
\smallskip
\Large Scientific Software Center\\
\smallskip
}

\vspace{3cm}

\textbf{\large Bachelor Thesis}

\vspace{0.5\baselineskip}
{\huge
\textbf{Viability Evaluation of Homomorphic Encryption in Machine Learning}
}
\end{center}

\vfill

{\large
\begin{tabular}[l]{ll}
Name: & Julian Partanen\\
Matriculation Number: & 4204806\\
Supervisor: & Dr. Dominic Kempf\\
Submission Date: & December 02, 2025
\end{tabular}
}

\end{titlepage}

% Add a brief summary of your topic and contributions (Zusammenfassung) in German *and* in English:
\vspace*{\fill}
\begin{center}
  \large
    \textbf{Abstract}\\
  - \textit{English} -
\end{center}


\begin{flushleft}
\setlength{\leftskip}{1cm}
\setlength{\rightskip}{1cm}
The abstract has to be given in German \textbf{and} English. It should
be between half a page and one page in length. It should cover in a
readable and comprehensive style the context of the thesis, the
problem setting, the objectives, and the methods developed in this
thesis as well as key insights and results.


\end{flushleft}
\vspace*{\fill}

\vspace*{\fill}
\begin{center}
  \large
    \textbf{Zusammenfassung}\\
  - \textit{Deutsch} -
\end{center}


\begin{flushleft}
\setlength{\leftskip}{1cm}
\setlength{\rightskip}{1cm}
Maschinelles Lernen (\acs{ml}) and künstliche Intelligenz (KI) werden in immer mehr Anwendungsbereichen eingesetzt.
In einigen Umgebungen wirft die Integration dieser Technologien über Dittanbieter- oder Cloud-Dienste jedoch erhebliche Bedenken hinsichtlich des Datenschutzes auf.
Volle homomorphische Verschlüsselung (\acs{fhe}) ermöglicht es \acs{ml}-Modellen mit verschlüsselten Daten zu arbeiten, sodass sensible Nutzungen dieser Modelle ausgelagert werden können, ohne dass private Daten offengelegt werden.

Diese Arbeit untersucht, ob \acs{fhe} dieses Versprechen erfüllt, und ob es derzeit für echte \acs{ml}-Anwendungen geeignet ist.
Zu diesem Zweck haben wir eine Reihe von Experimenten durchgeführt, in denen verschiedene \acs{fhe}-Modelle miteinander, und mit ihren Klartext-Entsprechungen, verglichen wurden.
Wir haben diese Vergleiche für mehrere Datensätze und Klassifizierungsaufgaben durchgeführt, sowohl für verschlüsselte Inferenz als auch Training.
Als einen etwas anspruchsvolleren Anwendungsfall haben wir außerdem zwei maßgeschneiderte Ansätze entwickelt, um die Aufgabe der Erkennung benannter Entitäten (\acs{ner}) homomorph zu lösen.
Darüber hinaus geben wir eine zugängliche Einführung in die theoretischen Grundlagen von \acs{fhe}, um das Verständnis und die Erklärung unserer Resultate zu unterstützen.

Unsere Ergebnisse zeigen, dass \acs{fhe} zwar in eine Reihe von \acs{ml}-Aufgaben wie z.B. die Spam-Erkennung integriert werden kann, jedoch weiterhin erhebliche Limitierungen bestehen.
Dazu gehören ein beträchtlicher Mehraufwand in der Laufzeit sowohl auf Client- als auch auf Serverseite, sowie eine verringerte Genauigkeit bei einigen Modellen.
Angesichts dieser Einschränkungen diskutieren wir auch alternative Ansätze zur Gewährleistung des Datenschutzes im \acs{ml}, wie beispielsweise eine clientseitige Modellausführung.

Das Ziel dieser Arbeit ist es, sowohl \acs{fhe} als Datenschutztechnologie einzuführen, als auch eine informierte Entscheidung darüber zu ermöglichen, ob \acs{fhe} für eine bestimme \acs{ml}-Anwendung geeignet ist, oder ob alternative Techniken möglicherweise eine passendere Lösung darstellen.
\end{flushleft}
\vspace*{\fill}

\acresetall


% MAIN PART:
% Table of contents (Inhaltsverzeichnis)
\tableofcontents
\cleardoublepage
\pagenumbering{arabic}

%%%%%%%%%%%%%%%%%%%%%%%%%%%%%%%%%%%%%%%%%%%%%%%%%%%%%%%%%%%%%%%
% Here, the actual content of your thesis begins
% You can either put all the text here or use individual files to store the chapters of your thesis.
% Below are templates for both alternatives.

\chapter{Introduction}\label{intro}

\section{Motivation}

Especially in recent years, usage of \ac{ai} and other \ac{ml} tools has become more and more widespread across many industries and applications.
\acp{llm} have become a very accepted by the general population as an alternative to search engines and are being used as a tool for a large variety of tasks, many of which necessitate users revealing a lot of sensitive information about themselves to a much larger scale than they had to when using search engines.
Furthermore many industries that inherently deal with very sensitive data could highly benefit from the use of \ac{ml} tools to solve common problems.
For example in many medical fields \ac{ml} tools could enhance a doctors diagnostics and treatment abilities, for which they would however require access to highly sensitive medical records of patients.

On the other hand, these \ac{ai} and \ac{ml} tools can be very computationally demanding and quite difficult to set up.
As a result, for integrating with these tools into a new application there is currently a strong preference towards using existing setups by some external cloud provider through a public \acs{api}, or relying on some external service provider in some other way.
This begs the question of how data privacy, especially for these more sensitive applications, can still be guaranteed.
Since regulations rightfully prohibit the sharing of for example medical records with third party service providers, the result was often a lack of adoption of current \ac{ml} tools into these fields.

\section{Introducing \acl{fhe}}

In this thesis we will discuss a possible solution to this problematic, called \acf{fhe}.
\ac{fhe} allows a client to still use an external service provider or server for their compute or storage resources without having to entrust any of their data to this provider.
This is achieved by encrypting the data on the client before sending it to the server with the server maintaining its ability to perform computations on this data.
After applying the required computations, the server sends the still encrypted result back to the client who can then decrypt it.
In essence, the encryption and decryption become homomorphic towards any function $f$ that the server applies on the ciphertext.

\begin{figure}
    \begin{center}
        \includegraphics[width=0.5\textwidth]{../figures/fhe_sequence_diagram.pdf}
    \end{center}
    \caption{A sequence diagram showing a generic client-server \ac{fhe} interaction}\label{fig:fhe_sequence_diagram}
\end{figure}

A generic application of \ac{fhe} can be seen in \cref{fig:fhe_sequence_diagram}.
Alice (the client) encrypts some sensitive input data $x$ (step 1), and sends it to Bob (the server) in step 2.
Bob has no way to derive $x$ from its decryption however it can still apply a function $f$ on the encrypted message (step 3).
Due to the homomorphic property of the encryption scheme this is equivalent to having an encryption of the result of $f$, and Bob also cannot learn anything about this result (or any intermediate value) of the computation he performed.
Afterwards Bob sends the encrypted result back to Alice (step 4), without revealing anything about the nature of $f$ to Alice.
Alice decrypts the received message, yielding the correct result $f(x)$ of the computation in step 5.
During this whole interaction, Alice maintained her goal of not revealing $x$, $f(x)$, or any intermediate values to Bob at any point, while Bob maintained his goal of never revealing $f$ or any of its properties to Alice.

Applied to \ac{ml}, this could allow us to benefit from the compute resources and ready-to-use deployments of cloud infrastructure without the drawbacks to data privacy that come with sharing our data with the cloud provider.

We want to make sure to differentiate \ac{fhe} from some other privacy-improving innovations in the field of \ac{ml}:
\begin{itemize}
    \item \emph{\ac{dp}} is a technique employed to make sure that a \ac{ml} model does not learn private information that only occur ones in the training (e.g. names or phone numbers). The goal is to make sure that rare secrets in the training data cannot be leaked to users that are later-on using the model.
    \item \emph{\ac{fl}} allows multiple parties a train one shared \ac{ml} model without having to reveal their respective training datasets to one another. The goal is to train a model on a larger, shared dataset my consolidating data from multiple parties without these parties having to reveal their data to each other.
\end{itemize}
While these are also very interesting technologies, it is important to note that they solve different problems than \ac{fhe}.

Another use case is \emph{\ac{mcp}} which allows multiple parties to encrypt their respective inputs and compute a shared function on it.
While this is often achieved by employing \ac{fhe}, we will ignore this use case in this thesis and solely focus on \ac{fhe} allowing us to use a third-party for compute without sharing our data with them and without this third-party having to share their model architecture and weights with us (as described above and in \cref{fig:fhe_sequence_diagram}).

\section{Goals}

The core concept of homomorphic encryption has been around for some time now, first introduced by the work \citetitle*{rivest_data_1978} by \citeauthor{rivest_data_1978} in 1978 \cite{rivest_data_1978}.
While there have been some encryption schemes like \acs{rsa} that featured some kind of homomorphism (homomorphic in multiplication in the case of \acs{rsa}), it took until 2009 and the introduction of the Bootstrapping algorithm by \citeauthor{gentry_fully_2009} to truly deliver on the promise of \ac{fhe} \cite{gentry_fully_2009}.
In the years after that many \ac{fhe} schemes have been introduced, with increasing improvements in runtime and general usability.

As a result \ac{fhe} seems to have turned from a theoretical possibility into a practical solution for some real-world applications.
A prominent examples for this is Apple adopting \ac{fhe} for image recognition tasks on the iPhone \cite{noauthor_combining_nodate}.
Applying \ac{fhe} to \ac{ml} models has especially been a topic of interest in recent publications and research, some even achieving to run \acp{llm} homomorphically.

In this thesis we want to investigate some of these promises and find out if and when \ac{fhe} can and should be used for real-world \ac{ml} tasks.
For this we developed a suite of benchmarks to measure how \ac{fhe} execution impacts various \ac{ml} models running on a variety of datasets and solving a variety of tasks.
We will mostly focus on the \emph{concrete-ml} \cite{noauthor_concrete_2025} framework that seems to be on the forefront of providing easy compilation of \ac{ml} models into \ac{fhe} equivalents.
In contrary to the frameworks own benchmarks (provided as Jupyter Notebooks in their GitHub repository) we aim to get a more complete view of the frameworks practical usability and viability while focusing more on real-world usage scenarios.

\chapter{Background}

The topic of this thesis falls into an overlap between the fields of cryptography and machine learning, and as such we will first introduce some concepts from both fields in this chapter. We will not assume anything more than surface-level knowledge in any of the two fields from the reader.

\section{Cryptography Concepts and FHE}

\subsection{Basic Notation and Phrases}

Throughout this work we will use some of the following notations and phrases that are also commonly encountered in other cryptography-related works. We will introduce these concepts informally aiming for a basic understanding of them while linking to formal definitions.

First we need to introduce the concept of randomness in the context of computational theory. For this we will use the definitions of Anjeev Arora's and Boaz Barak's "Computational Complexity: A Modern Approach" \cite{arora_computational_2009}. Many algorithms in Cryptography rely inherently on randomness and thus cannot be modeled by a standard deterministic Turing Machine (TM) with polynomial programs. Instead we need to model their computation using the Probabilistic Turing Machine (PTM) which is an extension of the TM. In contrast to the TM which only has one transition function $\delta$, the PTM has two transition functions $\delta_0$ and $\delta_1$ and chooses at each step at random which one to apply, with probability half to apply $\delta_0$ and half to apply $\delta_1$.

Similarly how $\mathbf{P}$ is the class of decision problems that are solvable in polynomial time by a TM, $\mathbf{BPP}$ is the class of decision problems that are solvable in polynomial time by a PTM. Note that since the TM is a special case of the PTM (the one where $\delta_0 = \delta_1$), and since it is possible to simulate all branches of a PTM with a TM in time $2^{poly(n)}$ it holds that $\mathbf{P} \subseteq \mathbf{BPP} \subseteq \mathbf{EXP}$ (see chapter 7.1 in \cite{arora_computational_2009}).

An alternative definition of a PTM would be to add a second tape to a TM which includes bits that are results of fair coin tosses, i.e. each symbol on the second tape is chosen mutually independent and uniformly at random from the set $\{ 0, 1 \}$, and then the tape is fed to a TM in addition to the regular input tape. From this viewpoint comes a phrasing of referring to the randomness in cryptographic algorithms like encrypt or key generation functions as 'coins'. These independent random coin-flips can be viewed as an implicit input to any such algorithm, and are in practice most commonly obtained by pseudo-random number generators.

Furthermore, as introduced in the lecture notes of Mihir Bellare and Phillip Rogaway \cite{bellare_introduction_2005} in chapter 1.3, we will use the following notation to mean that $i$ is a value chosen uniformly at random from the set $\mathcal{S}$: $i \xleftarrow{\$} \mathcal{S}$. This notation can also be used for algorithms, e.g. $i \xleftarrow{\$} \text{Encrypt}$ which can be viewed as $i$ being a random value from the image of 'Encrypt', or a random result of all possible branches of the PTM-equivalent of the 'Encrypt' algorithm chosen uniformly at random. The \$ denotes the random coins used as input. Note that some literature may also use $R$ instead of \$.

\subsection{A First Somewhat Homomorphic Encryption Scheme}

We will begin by introducing a simple version of an encryption scheme described by Dijk, Gentry, Halevi and Vaikuntanathan in 'Fully Homomorphic Encryption over the Integers' \cite{van_dijk_fully_2010} as an easy to understand example to explain some fundamentals of FHE on before moving on to more complicated and current schemes.

This first scheme encrypts bits, so our plaintext space is defined as $m \in \{ 0, 1 \}$. Our secret key $p$ is an odd integer that is sufficiently large.

We will now encrypt our $m$ into our ciphertext $c$:

\begin{equation}
    c \leftarrow p q + 2 r + m
    \label{eq:dijk_encryption}
\end{equation}

Here $q$ and $r$ are also some integers $\neq p$, sampled at random from some intervals, with $r$ being much smaller than $p$ and $q$.

The ciphertext $c$ can then be decrypted by doing the following:

\begin{equation}
    m \leftarrow \left( c \mod p \right) \mod 2
\end{equation}

This encryption scheme is correct because by calculating $\mod p$ we get rid of the $p \cdot q$ term in \ref{eq:dijk_encryption} while $\mod 2$ nullifies $2r$, with our $m$ staying untouched since it is either 0 or 1 and thus smaller than 2. Effectively we encode our message in an integer by making it even if it is 0, and odd if it is 1, and then we add $q$ times our secret $p$ to it in order in obfuscate the message. $p$ is odd, but $q$ can either be odd or even making the result unpredictable.

\emph{Remark} This is also the reason why $p$ must be odd, if $p$ where even then $pq$ would always be even as well, making the resulting ciphertext even if $m$ is 0, and odd if $m$ is 1, and thus providing no security at all.

To see why this encryption is secure we first reduce the encryption to the following problem: Given polynomially-many samples (in the security parameter $\lambda$)
\begin{align}
    x_1 &= p q_1 + r_1 \\
    x_2 &= p q_2 + r_2 \\
    ... &\\
    x_n &= p q_n + r_n
\end{align}
for a randomly chosen odd integer $p$, find $p$. So in other words the challenger would generate the parameters and then provide the adversary only with $x_1$ to $x_n$, who then would have to find $p$. Compared to our encryption in \ref{eq:dijk_encryption}, the $x_i$ would be many different ciphertexts, and the $r_i$ would be results of $2r + m$.

While this looks very similar to the well-known Greatest Common Divisor (GCD) problem, the key difference here is the random noise $r_i$ which makes the results only approximates of common divisors of $p$. Because of that this variation is called \emph{Approximate} GCD problem (AGCD). While the GCD problem can be efficiently solved using Euclid's algorithm, AGCD with correctly chosen parameters is hard to solve. While we leave the proof of that to \cite{van_dijk_fully_2010}, intuitively this is the case because when applying Euclid's algorithm the noise values $r_i$ amplify each other rendering the result meaningless.

Finally we will cover the most interesting attribute of this encryption scheme: As long as $r$ stays sufficiently smaller than $p$, multiple ciphertexts obtained from \ref{eq:dijk_encryption} are homomorphic in both addition and multiplication which can easily be shown:

\begin{align}
    c_1 + c_2 &= pq + 2 r_1 + m_1 + pq + 2 r_2 + m_2 \\
              &=2 pq + 2 \left( r_1 + r_2 \right) + m_1 + m_2
    \label{eq:dijk_hom_add}
\end{align}

Decryption would first remove the first term with the $\mod p$ operation, and then remove the second term with the $\mod 2$, only leaving $m_1 + m_2$.

\begin{align}
    c_1 \cdot c_2 &= \left(pq + 2r_1 + m_1 \right) \left( pq + 2r_2 + m_2 \right) \\
                  &= p^2 q^2 + 2 r_2 pq + pqm_2 + 2r_1pq + 4r_1r_2 + 2r_1m_2 + pqm_1 + 2r_1m_1 + m_1m_2\\
                  &= p\left(pq^2 + 2r_2q + qm_2 + 2r_1q + qm_1\right) + 2\left( 2r_1 r_2 + r_1m_2 + r_1 m_1 \right) + m_1 m_2
    \label{eq:dijk_hom_mult}
\end{align}

Once again decryption would remove the first term with the $\mod p$ operation, and then the second term with the $\mod 2$ operation, only leaving $m_1 m_2$.

Please note that because of the $\mod 2$ operation, in the case of $m_1 = m_2 = 1$, $m_1 + m_2$ becomes $0$ which is fine since our message space is binary anyway. So when only looking at the plaintext effectively addition is the XOR-operator and multiplication is the AND-operator.

We can also see some deficits of this simple homomorphic encryption scheme: The noise $r$ grows linearly during addition (see the $r_1 + r_2$ term in \ref{eq:dijk_hom_add}), and quadratically during multiplication ($r_1r_2$ in \ref{eq:dijk_hom_mult}). Since these homomorphic evaluations only stay correct while the noise is sufficiently smaller than $p$ as already stated, we can only homomorphically evaluate a limited amount of additions and multiplications before decryption would output an incorrect result. This is why Gentry called these kind of schemes \emph{somewhat} homomorphic encryption scheme in \cite{gentry_fully_2009}, while also providing us with an general algorithm to turn somewhat homomorphic schemes into fully homomorphic ones which we will look at next.

Another deficiency of this scheme is that the ciphertext grows in size during homomorphic evaluation, e.g. after multiplication the 'key-part' of the ciphertext grew quadratically (the $q^2 p^2$ term in \ref{eq:dijk_hom_mult}). In other words this scheme doesn't compactly evaluate circuits, as defined by Gantry in definition 2.1.2 and 2.1.3 \cite{gentry_fully_2009}. Compact Homomorphic Encryption requires the decryption complexity to not by dependent on the circuit depth which in turn requires the ciphertexts and keys (which are the inputs of the decryption algorithm) to not grow with the depth of the circuit.

While this simple scheme is not representative of modern FHE schemes, it helps to understand the goals as well as struggles of FHE in general. Most encryption schemes introduce some sort of noise parameter that is strictly required for the security of the scheme, but also introduces limitations regarding accuracy and/or depth of homomorphic evaluation. Balancing security with these drawbacks as well as handling noise in ciphertexts remain current challenges even with state-of-the-art FHE schemes.

\section{Machine Learning Concepts and Model Types}

\chapter{Method}

In this chapter we will describe the experiment setup used to measure various trade-offs of homomorphic encryption in different machine learning scenarios.
We will describe the technologies and general methodology as well as all the different datasets and model architectures used.

\section{Technological Overview}

The source code of all experiments described in this thesis is open-source and available on GitHub \cite{partanen_julianfpfhe-experiments_2025}.
The repository includes a README file with some basic instructions in how to run the experiments.
This means that the reader may generate every data point or graph in this thesis themselves.
The absolute numbers will differ because of hardware and software environment differences, but the relative numbers as well as general findings should thus be reproducible by anyone.

The experiments were implemented in Python, the de facto standard language in the field of machine learning.
We have added astral's \emph{uv} \cite{astral_uv_nodate} as a project and dependency management tool to make running the script with all necessary dependencies easier and to pinpoint all dependencies to a specific version to increase reproducibility.

We defined common interfaces for all experiments as well as datasets to ease their implementation, reduce code duplication, and to ensure to only change desired attributes between different experiments to increase comparability.

For generating result data we used Python's statistics and csv modules as well as \emph{matplotlib} \cite{noauthor_matplotlib_nodate} to generate all graphics.
The graph generation runs independently from the experiment execution which allows redrawing graphs of existing experiment results to for example change the styling of graphs after the fact.

The python package also uses \emph{click} \cite{pallets_welcome_nodate} to add an easy-to-use \acs{cli} to be able to adjust parameters like execution repetitions, which experiment-dataset combinations should be executed, or which graphs should be generated.

\section{The Examined \acs{fhe} Framework: concrete-ml}

We chose the \emph{concrete-ml} framework to implement a variety of machine learning models in \ac{fhe} \cite{noauthor_concrete_2025}.
It is built on top of the \ac{tfhe} compiler \emph{concrete} which turns Python programs into \ac{tfhe} circuit-equivalents \cite{noauthor_concrete_2025-1}.
Both projects are fully open-source and are being developed by the cryptography start-up Zama.
While concrete-ml and Zama are not the only ones trying to implement \ac{fhe} for machine learning applications, from what we have seen concrete-ml seems to be by far the easiest, most complete, and most high-level solution at the time of writing.
It's main selling points are support for a large variety of models, including custom models, while requiring very little or no cryptography knowledge from the developer implementing these models.
This means that data scientists and in general people with some machine learning experience but no \ac{fhe} or cryptography knowledge should be able to build \ac{fhe} models with concrete-ml without leaving the comfort of Python and Pytorch or scikit-learn.
We will look at some alternatives to concrete-ml later, but during our experiments we exclusively relied on concrete-ml.

\subsection{Feature Overview}

We already explained the \ac{tfhe} scheme to some extend in \cref{TFHE}.
Concrete largely implements this scheme as described in that section, with some extensions to increase it's functionality and flexibility.
It's main innovation was \emph{\acf{pbs}} (which we already mentioned in \cref{TFHE_building_blocks}) which allows to execute functions that can be expressed as a look-up table during the Bootstrapping process.
This \ac{pbs} approach is used a lot by concrete and concrete-ml, in practice it implements most non-linear operations (i.e. everything that cannot be expressed as addition and multiplication with a constant).
When looking at a neural network, many operations are linear like matrix multiplications with the model weights.
However machine learning models have to incorporate non-linear functions to be able to separate non-linear data which means that in order to run these models homomorphically we need to be able to evaluate these non-linear functions in \ac{fhe} as well.
For neural networks, these functions mostly are the activation functions, and in concrete-ml they are mostly implemented using look-up tables during \ac{pbs} \cite{dolev_programmable_2021} \cite{noauthor_concrete_2025-1}.

We will now describe the high-level feature set of concrete-ml and some real-world applications they might proof useful in:

concrete-ml provides 21 Built-In model classes which aim to be direct replacements for their scikit-learn counterparts and also provide the same scikit-learn interfaces.
12 of these models are simple linear models (e.g. logistic regression with different training methods, see \cref{LogisticRegression}), 6 of them are \acp{cart} (see \cref{CART}), 2 of them are \acp{fnn} (see \cref{FNN}), and there also exists one model class for \ac{knn} classification (see \cref{KNN}).
For our experiments we chose one model class from each category to evaluate as much of concrete-ml's feature set as possible without having to test all 21 models.
Some of these model classes provide a \texttt{from\_sklearn\_model} method to initialize the model object from a pre-trained scikit-learn model, others however do not (most notably the \ac{fnn} and \ac{knn} models).
These models need to be trained from the ground up using concrete-ml's model class and \texttt{fit} method \cite{noauthor_concrete_2025-1} \cite{noauthor_concrete-mluse_case_examples_nodate}.

If this is too restricting regarding model architecture and/or re-training requirements concrete-ml also supports custom PyTorch or \ac{onnx} models.
For our experiments we disregarded the \ac{onnx} option and went with custom PyTorch models instead because of PyTorch's overwhelming popularity.
concrete-ml allows to turn a custom self-defined PyTorch model into an \ac{fhe} either before or after training, allowing us to deploy already fully trained neural networks into an \ac{fhe} setting \cite{noauthor_concrete_2025-1} \cite{noauthor_concrete-mluse_case_examples_nodate}.

Finally concrete-ml also promises support for more modern and complex model architectures as well, most notably transformer models like the ones being used in \ac{llm}.
In addition to evaluating these models fully in \ac{fhe}, concrete-ml also allows for a hybrid deployment, where the linear layers are executed homomorphically on the server, and the non-linear activation functions in clear on the client, with the intermediate results being exchanged and de-/encrypted in between.
concrete-ml provides some use-case examples that include running \acs{gpt}-2 completely homomorphically.
While this looks very exciting, we did not include it in our suite of experiments for this thesis because of runtime concerns.
We aimed to make our benchmarks as close to real-world examples as possible which for a \ac{llm} always means generating a considerable amount of tokens for a sizeable amount of input tokens.
Zama's own benchmarks however already show that predicting the next token after 8 input tokens took almost 3 minutes while using a single attention head, and over 14 minutes while utilizing 12 attention heads.
Since every following token would have to take the previously generated tokens into consideration, this runtime would only increase for every following token resulting in unbearable runtimes for larger sets of input/output tokens.
We struggled to come up with real-world use-cases that would strictly require an \ac{llm} while not requiring more than single digit input/output tokens \cite{noauthor_concrete_2025-1} \cite{noauthor_concrete-mluse_case_examples_nodate}.

\subsection{Pre- and Post-Processing}

Our experiment results will include runtimes for \ac{fhe} pre-/post-processing in addition to the processing runtimes.
We introduced this terminology to capture all operations that need to be executed on the client-side before the actual \ac{fhe} circuit can be applied on the inputs, or after the output of the \ac{fhe} circuit can be interpreted.
This of course includes encryption of the inputs during pre-processing, and decryption of the output during post-processing.
However it also includes a couple of other operations as well, most notably quantization and serialization during pre-processing, and dequantization and deserialization during post-processing which we will explain in the following.

We already covered quantization in \cref{quantization}.
Specifically for \ac{ml} applications and in concrete-ml, quantization has to be done with both inputs and model weights since they all are almost always floating point values.
The good aspect of this however is that in \ac{ml} the lost precision does often not result in false computations, but just in minor differences in output probabilities.
However if we already are close to the decision boundary, for classification problems this can mean more false classifications, and also regression models will have more inaccuracies due to quantization.
How much a model accuracy is affected by quantization heavily depends on the type of model in use and how the input data is shaped.
For example linear models tend to suffer a lot less from quantization because the error introduced by it is not accumulated across many operations or model layers.
In concrete-ml, quantization is also a lot less aggressive for linear models since these do not use look-up tables which cannot work with high prevision integers since they would drastically increase the size of these look-up tables (we already mentioned the effect of \ac{pbs} on quantization bit-width in \cref{quantization}) \cite{noauthor_concrete_2025} \cite{noauthor_concrete_2025-1}.

Specifically for neural networks, concrete-ml offers two types of quantization: \ac{ptq}, and \ac{qat}.
We will not explain these quantization algorithms in detail, it is however important to point out that, as the name suggests, \ac{ptq} can be applied after training while \ac{qat} executes during training introducing the requirement of re-training any given model from scratch.
\ac{qat} is on the other hand considered to return more optimal quantization resulting in less accuracy loss.
This is why, as already mentioned, some model types is concrete-ml do not allow to be imported directly from an already trained model but require pre-training (e.g. the built-in NeuralNet classes) \cite{noauthor_concrete_2025}.

It is worth noting that quantizing \ac{ml} models is a common practice even outside the \ac{fhe} space, mostly with the goal to reduce model sizes and thus hardware requirements for running them.
concrete-ml uses the Brevitas library for performing \ac{qat} on neural networks which is not an \ac{fhe}-specific library \cite{noauthor_concrete_2025} \cite{noauthor_xilinxbrevitas_2025}.
However in \ac{fhe} and for non-linear models like neural networks, this quantization must be quite aggressive since concrete also has limits to what bit-widths it can support.
Because of this we will always measure accuracy and F1 scores during our experiments to see how big of an impact the quantization has on any given models accuracy on each dataset.

After quantization and encryption, the client has to also perform serialization.
This is the step of encoding our data into a common format that can be send over the network to the server and includes all the required metadata so that the server can read and interpret the ciphertexts correctly.
These three steps also have to be reversed on the output returned by the server in reverse order (post-processing) \cite{noauthor_concrete_2025} \cite{noauthor_concrete_2025-1}.

\subsection{Setup and Usage}

The most prominent problem that concrete-ml tries to solve is encrypted inference, i.e. we deploy an already trained \ac{ml} model on a relatively powerful server machine, and users can encrypt inputs, send them to the server, and receive encrypted outputs of the model without the server being able to decipher either of them.
For this we would use concrete-ml to compile a model into an \ac{fhe} equivalent model, and export that into two archives: A server.zip and a client.zip.
The first one contains the model and everything the server needs to run it homomorphically.
The second one contains cryptographic parameters and descriptions of all required pre-/post-processing that needs to be done by the client (but reveals no information about the model or it's weights).
An example for how this process looks can be seen in \cref{code:compile_model}.

\begin{lstlisting}[language=Python, caption=Simple example of compiling an already trained logistic regression model into the FHE equivalent, label=code:compile_model]
    # executed by the model developer in order to generate a server.zip and client.zip
    fhe_model = LogisticRegression.from_sklearn_model(model, example_input_data, n_bits=8)
    fhe_model.compile()
    dev = FHEModelDev(path_dir=<path where server.zip and client.zip will be saved>, model=fhe_model)
    dev.save()
\end{lstlisting}

The example\_input\_data is required so that concrete-ml can automatically figure out the best quantization parameters for the model.
It can be an excerpt from the training data, or some randomly generated data within the same boundaries as the training data.
After this the model can be deployed to the server that will be used to run it with encrypted user inputs.
An example of how a deployment of this can look can be seen in \cref{code:client_server_inference}.

\begin{lstlisting}[language=Python, caption=Client/Server example of encrypted inference, label=code:client_server_inference]
    # on the server-side: Init server object from a server.zip
    server = FHEModelServer(path_dir=<path to server.zip>)
    server.load()
    # send client.zip to our client

    # client-side: Init client object from received client.zip and encrypt input
    client = FHEModelClient(path_dir=<path to client.zip>)
    serialized_evaluation_keys = client.get_serialized_evaluation_keys()
    encrypted_input = client.quantize_encrypt_serialize(input)
    # send serialized_evaluation_keys and encrypted_input to the server

    # server-side: Run model inference in FHE on encrypted input
    encrypted_result = server.run(encrypted_input, serialized_evaluation_keys)
    # send encrypted_result back to our client

    # client-side: decrypt result
    result = client.client.deserialize_decrypt_dequantize(encrypted_result)
\end{lstlisting}

Please note that the serialized\_evaluation\_keys can be viewed as the public key of the client and importantly cannot be used by the server to decrypt the inputs.
They are needed for example for the bootstrapping process (for how the bootstrapping key is used see \cref{Bootstrapping}).
While concrete-ml also offers all-in-one functions to run inference on \ac{fhe} models homomorphically, in our experiments we always implemented this client-server setup to simulate how a real-world deployment as closely as possible.

\section{Experiments and Models}

With our experiments, we aim to find out how well concrete-ml can fulfill real-world use cases, especially while looking at the main two culprits of \ac{fhe} execution which are runtime performance and degraded model accuracy and drawing comparisons both between models and datasets as well as to regular clear executions of the same experiment.

To test the feature set promised by concrete-ml as exhaustively as possible, we implemented a variety of tests which generally can be split in three categories.

\subsection{Encrypted Inference with Built-In Models}\label{encrypted_inference}

Our first group of experiments tests the off-the-shelf built-in models provided by concrete-ml themselves on a set of simpler classification datasets across multiple real world problems like image and document classification (see section \cref{datasets}).
We expect these models to perform relatively well since they are purpose-built to be used in \ac{fhe}, architecturally simpler, and since they only need to solve relatively easy classification problems.
For every model/dataset combination in this category we follow the same basic steps:

\begin{enumerate}
    \item Train our model in clear on the train dataset
    \item Compile the trained clear model to the \ac{fhe} equivalent, or train the untrained \ac{fhe}-equivalent model on the same train dataset (depending on model in use)
    \item Run inference using the clear model on all samples in the test dataset while measuring the required time for completing this step
    \item Run the required \ac{fhe} pre-processing (quantization, encryption, serialization) on all the samples in the test dataset while measuring the required time for completing this step
    \item Run inference using the \ac{fhe} model on all encrypted test dataset samples while measuring the required time for completing this step
    \item Run the required \ac{fhe} post-processing (deserialization, decryption, dequantization, some last operation that needs to run in clear like an argmax) on all the encrypted results from the previous steps while measuring the required time for completing this step
    \item Return the four measured timings as well as accuracy and F1 scores for both clear and \ac{fhe} execution
\end{enumerate}

These steps are then repeated 10 times for each model/dataset combination (including the model training), and we compute the average values as well as standard deviations for all of the returned metrics across these 10 executions.

\subsubsection{Logistic Regression}\label{LogisticRegressionExp}

This model represents the set of linear models that concrete-ml has built-in support for.
We trained it using the LogisticRegression model from scikit-learn, and then used the \texttt{from\_sklearn\_model} method of the concrete-ml object to turn the trained model into it's \ac{fhe} equivalent, using a relatively high bit-width of 8.
This is possible because this  model is linear and does not require table look-ups.
In fact it is even implemented in a leveled way, meaning it should not require Bootstrapping at all.
The operations that are executed inside \ac{fhe} are just one matrix multiplication with the model weights and an addition with the model bias (see \cref{LogisticRegression} for a full explanation) \cite{noauthor_concrete_2025}.

Because of this we expect the runtime overheads to be relatively low compared to the other models, and the accuracy loss due to quantization almost non-existent.
However with it being a very simple and only a linear model, we also expect it to perform quite poorly on more complex problems that require non-linearity to solve them.

\subsubsection{XGB Tree-based Classifier}

This XGBClassifier class is our representative for \ac{cart} models, and it is concrete-ml's equivalent class to the XGBClassifier from the XGBoost library that implements the Boosting algorithm (see \cref{CART}).
We train the XGBoost model together with a standard scaler and \ac{pca} for dimensionality reduction in a scikit-learn pipeline as recommended by the concrete-ml documentation.
This is something that probably would not be necessary if executed in clear, but we found that it also did not change the models accuracy much so we chose to follow this recommendation which just improves \ac{fhe} execution times.
We then again compiled the model into the \ac{fhe} equivalent using the \texttt{from\_sklearn\_model} method \cite{noauthor_concrete_2025}.

As mentioned in \cref{no_control-flow}, \ac{tfhe} (and \ac{fhe} schemes in general) do not allow control flow statements that depend on the input, and thus the implementation of a decision tree where every node in the tree introduces a new branch becomes anything else than trivial.
The strategy for implementing \ac{cart} models used by Zama in concrete-ml is outlined in \citetitle*{frery_privacy-preserving_2023} \cite{frery_privacy-preserving_2023}.
The core strategy comes down to replacing conditions with a parallel evaluation of all branches, and to retrieve the correct result by looking up the decision path afterwards.
While this sounds very inefficient, it actually is not that uncommon even outside the \ac{fhe} space.
This is because the way this parallelized execution of all branches is achieved is by computing all branching decisions simultaneously using tensor operations.
This allows for acceleration with GPUs which in turn can significantly boost performance.
There are libraries like Microsoft's Hummingbird that aim to do exactly this in order to be able to accelerate these more classical machine learning models with tensor operations on GPUs \cite{noauthor_microsofthummingbird_2025}.

In addition to this parallel branch execution it is also required to train the \ac{cart} model using already quantized data so that it can learn the decisions for the correct integers.
Since this model also includes a mix of linear and non-linear operations implemented with \ac{pbs}, it also requires carefully chosen (partially through benchmarking) cryptographic parameters to reduce the quantization error to a minimum \cite{frery_privacy-preserving_2023}.
Nevertheless we were still able to perform the quantization with a bit-width of 8 bit, the same as used for Logistic Regression.
This high bit-width was probably only achievable by using the PCA and scaler operations that reduce the input dimensionality and thus the required FHE circuit depth. Both low quantization bit-width and aggressive usage of PCA reduce model accuracy, balancing the two might proof to be an interesting optimization opportunity.

Since we are running our experiments on CPU, we expect this to how a significant impact on performance, much more than with Logistic Regression.
On the other hand we also expect the \ac{cart} model to significantly outperform Logistic Regression since it is able to separate non-linearly, as long as the data is still clustered by class.
Due to the non-linear operations we expect some measurable decreases in accuracy, however still to a relatively low level compared since this model should not require very deep circuits.

\subsubsection{Neural Network Classifier}

The NeuralNetClassifier class implements a \ac{fnn} mimicking the scikit-learn class (as described in \cref{FNN}) and interface with the same name while in reality it's implemented with PyTorch.
For hyperparameters we used 3 hidden layers with the same dimensionality as the input features, the \ac{relu} activation functions, and 100 training epochs with a learning rate of $0.01$.
For this builtin-in model concrete-ml requires training of the FHE model instead of compiling a trained clear model \cite{noauthor_concrete_2025}.
For this we used the \texttt{fit\_benchmark} method that trains the FHE model together with the equivalent clear model for direct comparison between the two.

For quantization we left the bit-width on the default which is 3 bits for both model weights inputs/activations.
Concrete-ml also includes some model specific quantization optimizations like \texttt{power\_of\_two\_scaling} which sets the quantization scale to a power two, making some operations more efficient by turning them into a simple bit-shift. Coupled with the \ac{relu} activation function, this speeds up inference.
Quantization is also further optimized with the use of \ac{qat} which is the main reason why this model has to be trained from scratch \cite{noauthor_concrete_2025}.

Nevertheless this bit-width is still much lower than what we used with the previous two models which is why we expect some noteworthy accuracy reductions of the FHE model compared to the clear model here.
The required computations are also significantly higher than for the previous models and include applying matrix products as well as the non-linear \ac{relu} function for each of the 3+1 layers.
Therefore we expect this neural network model to both have the highest inference times in general as well as the highest relative increase of inference time in FHE compared to the clear model.

\subsection{Encrypted Training with Built-In Model}

This second experiment group tests the use case of encrypted training.
Here instead of training a model on clear data and then performing inference on encrypted data we already perform the training on encrypted data.
This might be interesting in fields where training data is very sparse and the only available data is confidential and cannot be shared with the entity performing the model training or machine learning service.
It allows a training-as-a-service deployment, where e.g. people working in medical fields can train models on confidential patient data on external infrastructure without exposing this data to this infrastructure provider.
Of course using non-sensitive training data should always be preferred over this FHE approach, but the nature of some applications preclude the existence of non-sensitive training (e.g. a model to find brain tumor needs to be trained on brain scans which cannot be anonymized).
This will be tested on the same datasets and thus use cases as \cref{encrypted_inference}, however the experiment steps and the measured attributes differ slightly:

\begin{enumerate}
    \item Train our model in clear on the train dataset while measuring the required time for completing this step
    \item Run the required \ac{fhe} pre-processing (quantization, encryption, serialization) on all the samples in the train dataset while measuring the required time for completing this step
    \item Train the untrained \ac{fhe}-equivalent model on all encrypted train dataset samples while measuring the required time for completing this step
    \item Run the required \ac{fhe} post-processing (deserialization, decryption, dequantization) on the trained model weights and bias while measuring the required time for completing this step
    \item Run inference using the model trained in clear on all samples in the test dataset
    \item Run inference using the model trained with \ac{fhe} on all samples of the same test dataset
    \item Return the four measured timings as well as accuracy and F1 scores for both models
\end{enumerate}

\subsubsection{\acs{sgd} Classifier}

Currently the only model that support being trained on encrypted data in the concrete-ml framework is the \acl{sgd} Classifier/Regressor. As described in \cref{LogisticRegression}, this is just a Logistic Regression model with a different training algorithm.

We fitted the FHE circuit with some randomly generated sample data so that it learns the boundaries of the input values for proper quantization.
Thus the model only ever saw the real training data in encrypted form.
We also limited the model to $50$ training iterations, initialized the weights and bias with zeros, and used a batch size of $8$.
Each batch as well as the start weights and bias got encrypted and passed to the model separately which means that FHE does not break the online learning properties of \ac{sgd}.
After training we decrypt the resulting weights and bias, and initialize a new SGDClassifier from scikit-learn with these weights and bias.

As with the Logistic Regression experiment (see \cref{LogisticRegressionExp}) we used a very high bit-width of $8$ for quantization.
We expect some accuracy loss due to the use of \ac{sgd} with limited iterations in training which might result in a model that is only approximately fitted optimally on the training data.
Accuracy losses due to quantization on the other hand should be very low.
Computing and applying gradients is a computationally quite heavy operation compared to the previous inference tasks.
Also this needs to be done on the training data set and not on the test data set which in our experiment suite usually contains five times as many samples (see \cref{datasets}).
As a result we expect the runtimes of this encrypted training experiment to be very high even though the resulting model architecture is very simple.

\subsection{Encrypted Inference with custom approaches for the \acl{ner} task}

Our third and final experiment group aims to push the boundaries of what is possible with \ac{ml} in \ac{fhe}.
While our previous experiments used relatively simple classification problems as use cases while mainly focusing on how \ac{fhe} impacts runtime performance and model accuracy, these experiments will try to scout out if \ac{fhe} is practical for more demanding use cases as well.

For this we chose the \ac{nlp} task \acf{ner} as a representative for a more demanding task requiring more complex model architectures.
As described in \cref{ner}, \ac{ner} can be a quite demanding task requiring the model to learn the semantics of the input as well as many different rules regarding the context of a token which necessitates a certain parameter size of the model.
At the same time \ac{ner} is not nearly as demanding as more currently relevant tasks like large language modeling done by \acp{llm} while still being somewhat similar since both tasks require modeling language to some extend.

The experiment steps are equivalent to the ones in \cref{encrypted_inference} yet with much more flexibility in difference in the details since we implemented two approaches that try to solve the \ac{ner} task very differently.

\subsubsection{Custom PyTorch model with Embedding layer}

At the core of this approach we coined the \ac{ner} task into a classification problem by hiding the semantic complexities behind a semantic word embedding (see \cref{semantic_embed}), and then solving this classification problem using a simple \ac{fnn}.
We solved the problem of context with the sliding window approach:
The model has to classify one token at a time, and receives a fixed number of tokens before and after this token. We settled with a window size of $5$, meaning the model receives the token that it should classify and two tokens before and after respectively.

To be able to feed our text input to a model, we turned each token into an index by first regularizing the tokens and then assigning integers to them in order of appearance in the training data.
The regularization step is necessary to avoid situations where the same token gets assigned two different integers because of slight differences (e.g. different capitalization, additional spacings, ...).
Because of this the regularization includes making all characters lower-case.
We did not need to perform tokenization since the dataset in use was already tokenized. Refer to \cref{conll} for more information about the dataset we used.

For cases where our model encounters a token it has never seen during training we created the '<UNK>' token. All unknown tokens in a sample will be replaced with this token before the sample gets converted to token indices.
To help the model learn how to handle samples that include his '<UNK>' token we randomly replaced input tokens with our unknown token during training (with probability $0.05$ for each token such that different tokens would get masked in every epoch).

To turn the training data into sliding window samples we also introduced the '<PAD>' token to the model.
If the entity in question is one of the first or last two tokens in a sample we still feed the token that should be classified as the center token and replace the empty positions with '<PAD>'.
For example if our input is "Paris is the capital of France." (our example from \cref{ner}), and the model should classify 'Paris', it gets the sliding window "<PAD> <PAD> Paris is the" as input.
We can already see some deficiencies with this sliding window approach here since the model often does not get enough context (the quite important "capital of France" is cut out).
Nevertheless we still wanted to start of with a relatively small window size of 5 to keep model complexity low for the \ac{fhe} execution, with the option to increase it in the future.

Since we converted all tokens to lower-case, and since we the model only receives token indices and not the actual tokens, we added two auxiliary features to give the model some attributes of each of the tokens in addition to their indices: Capitalization information and token lengths.
The capitalization information can distinguish between a token
\begin{itemize}
    \item which exclusively consists of capital letters
    \item which starts with a capital letter
    \item which contains a capital letter somewhere
    \item having no capital letters
\end{itemize}
, resulting in 4 capitalization classes.
The word length is just the number of characters in the token, capped at $20$ for regularization.

Our PyTorch model architecture looks like this:
The 15-dimensional feature vector (token index, capitalization class, and token length for each of the 5 tokens in our sliding window) then gets passed through an embedding layer that generates a 128-dimensional embedding vector.
This then gets passed through a regular \ac{fnn} with 3 hidden layers with the following dimensions: 128, 128, and 64.
After each of these layers we used the \ac{relu} activation function as well as a dropout layer with a dropout rate of $0.2$ to reduce overfitting during training.

When only looking at the PyTorch features in use, this custom model only differs from the built-in model in two points:
It utilizes the PyTorch Embedding and Dropout layers.
While concrete-ml does not support all PyTorch layers and features, it does support these two.
Good support for the Embedding layer is especially very crucial since it is by far the most essential layer in our model design and is also a very elementary building block for more complex model architectures like transformer models.
The Dropout layer is only active during training and thus has only a chance to cause problems in \ac{fhe} conversion when \ac{qat} is being used.

We implemented the conversion of the model to \ac{fhe} using both \ac{ptq} and \ac{qat} with a quantization bit-width of 5.
We performed \ac{ptq} by first training the PyTorch model in clear and then using the \texttt{compile\_torch\_model} function of concrete-ml on it.
For this we used some randomly generated sample data with the same minimal and maximal bounds as the actual dataset.
For the \ac{qat} variant of this experiment we used the Brevitas \cite{noauthor_xilinxbrevitas_2025} library to replace all PyTorch layers of the model with Brevitas equivalents, as well as an additional \texttt{QuantLinear} layer at the beginning.
However we were not able to get this \ac{qat} variant to run at all for reasons we will explain in our results.

\subsubsection{Transformer Embeddings with \acs{knn} Classifier}

\section{Datasets}\label{datasets}

We used a variation of datasets across different classification problems commonly solved with a machine learning approach.
Some of these datasets are synthetic for maximum control over their characteristics to specifically measure a models behavior under certain conditions, while others are real-world datasets that we used to measure how well the models perform solving a variety of real-world problems, mimicking a production deployment of a homomorphic machine learning setup as much as possible.

All datasets use single-precision floating point numbers to store feature vectors to make them compatible with all concrete-ml models since some of these models don't support double-precision for their training data.
We made this choice to increase comparability both between \ac{fhe} and non-\ac{fhe} executions as well as across different model architectures and since the accuracy trade-off should be almost non-existing.

\subsection{XOR problem}

We handcrafted this tiny dataset for code testing with faster iterations as well as quick experiment runs, however it proofed to also be a good metric for how a model handles non-linearity in a dataset, and if a model is able to fit to problem with a very small training set.

\begin{figure}[h]
    \begin{center}
        \includegraphics[width=0.65\textwidth]{../figures/'XOR problem' dataset.png}
    \end{center}
    \caption{A plot of the feature space of the 'XOR problem' dataset, including both train and test sets}
    \label{fig:xor_problem_plot}
\end{figure}

The dataset's feature vectors are 2-dimensional.
The label is 1 if and only if one of the features is closer to 0.75 than to 0.25 while the other is closer to 0.25 than to 0.75.
Conversely if both features are approximately the same then the label is 0.

As the name suggests this mimics the boolean XOR operator, with some differences: The inputs are 0.25 and 0.75 instead of 0 and 1 to keep them between 0 and 1 even with noise applied.
Since some models require more than just 4 samples we also repeated each of them 10 times for both the training and test set resulting in a total of 40 samples each.
To make these repeated samples unique and to add meaningful differences between the training and test set we added some noise to all samples in the form of a random number drawn from an unique distribution between $\pm 0.24$.

The result can be seen in \cref{fig:xor_problem_plot}: A feature space with 4 feature clusters, arranged in a way that makes it impossible for a linear model to separate with an accuracy higher than 75\%.
This makes this dataset great in ensuring that more complex non-linear model actually come with an accuracy improvement.

\subsection{Iris}

\subsection{Digits}

\subsection{Breast Cancer}

\subsection{Synthetic}

This dataset is a generated at random using scikit-learn's \emph{make\_classification} function.
It includes 250 samples, of which 150 are used for training, and 100 for testing.
Both classes have only one cluster, making it linearly separable and thus even very simple models should have no problem achieving high accuracy scores on this dataset.

\begin{figure}[h]
    \centering
    \begin{subfigure}[t]{0.32\textwidth}
        \centering
        \includegraphics[width=0.95\textwidth]{../figures/'Synthetic, 50 features' dataset - with PCA applied.png}
        \caption{50-dim}
    \end{subfigure}
    \begin{subfigure}[t]{0.32\textwidth}
        \centering
        \includegraphics[width=0.95\textwidth]{../figures/'Synthetic, 500 features' dataset - with PCA applied.png}
        \caption{500-dim}
    \end{subfigure}
    \begin{subfigure}[t]{0.32\textwidth}
        \centering
        \includegraphics[width=0.95\textwidth]{../figures/'Synthetic, 5000 features' dataset - with PCA applied.png}
        \caption{5000-dim}
    \end{subfigure}
    \caption{Plots showing the feature space of three of our seven variations of the 'Synthetic' dataset, \ac{pca}-reduced.}
    \label{fig:synthetic_plot}
\end{figure}

The amount of features in this dataset is variable which allows us to specifically isolate the impact of feature size on model runtime and performance.
We start with 50-dimensional features, and gradually increase the dimensionality up to 5000, resulting in a total of 7 different variations of this dataset.
For each variation, only 10\% of the features are actually informative of the sample's label, while the other 90\% are just generated at random independently from the true labels.
This mainly tests a models resistance to overfitting.

The result can be seen in \cref{fig:synthetic_plot}.
Please note that in contrast to the 2-dimensional 'XOR problem' dataset we had to map these high dimensional features to 2 dimensions in order to plot them.
For this we used scikit-learn's \acf{pca} implementation.
Please note that by doing this, the visual representation of the dataset isn't accurate, for example in the plots it doesn't appear to be separable even though in reality it is.
The high amount of uninformative dimensions contributes to this effect.

\subsection{SMS Spam}

This is our first real-world dataset and we will use it to represent the use-case of document classification during our experiments.
The SMS Spam dataset was developed by \citeauthor{almeida_contributions_2011} \cite{almeida_contributions_2011} and consists of text messages that are labeled as either ham (label '0') or spam (label '1').

\begin{figure}[h]
    \centering
    \begin{subfigure}[t]{0.32\textwidth}
        \centering
        \includegraphics[width=0.95\textwidth]{../figures/'SMS Spam, 50 features' dataset - with PCA applied.png}
        \caption{50-dim}
    \end{subfigure}
    \begin{subfigure}[t]{0.32\textwidth}
        \centering
        \includegraphics[width=0.95\textwidth]{../figures/'SMS Spam, 2500 features' dataset - with PCA applied.png}
        \caption{2500-dim}
    \end{subfigure}
    \begin{subfigure}[t]{0.32\textwidth}
        \centering
        \includegraphics[width=0.95\textwidth]{../figures/'SMS Spam, all features' dataset - with PCA applied.png}
        \caption{All features}
    \end{subfigure}
    \caption{Plots showing the feature space of three of our eight variations of the 'SMS Spam' dataset, \ac{pca}-reduced.}
    \label{fig:sms_spam_plot}
\end{figure}

To extract feature vectors from the documents we use scikit-learn's \ac{tfidf} Vectorizer while filtering English stop words.
Similarly to the 'Synthetic' dataset we also have different variations of this dataset with different feature dimensions ranging from 50 to 5000, as well as one variation which includes all 7463 features.
Limiting the amount of features was achieved by ordering the features by term frequency across the corpus and only considering the top x.

Like with the 'Synthetic' dataset we also performed a 60/40 train/test set split which resulted in a dataset size of 3344 and 2230 samples for the training and testing sets respectively.

The result can be seen in \cref{fig:sms_spam_plot}.
The same limitation regarding \ac{pca} as mentioned for the 'Synthetic' dataset applies here as well.

\subsection{CleanCoNLL}\label{conll}

\chapter{Results}\label{chap:results}

We executed the described experiment suite on two different machines to get a more complete picture of \ac{fhe} runtime behavior:
\begin{itemize}
    \item Our working groups cluster with 2x Intel Xeon Gold 6230R CPUs (in total 52 cores/104 threads @2.1GHz, up to 4.0GHz) with 192GB of memory, representing a machine with very high multi-core but lower single-core performance
    \item A personal computer with 1x AMD Ryzen 5 7600 CPU (6 cores/12 threads @3.8GHz, up to 5.1GHz) with 32GB of memory, representing a machine with much lower multi-core but higher single-core performance
\end{itemize}

A lot of \ac{fhe} operations in concrete are very well parallelized while our clear executions of the same models are not.
Consequently we wanted to test the runtime on two configurations, one being more fair for the comparison between \ac{fhe} and clear executions since the \ac{fhe} execution cannot gain that much by parallelization on a small core-count CPU, and one representing a more realistic deployment for \ac{fhe} applications since server and cloud machines tend to have high core counts but smaller single core performance.
We also want to point out that the cluster was being used by other people as well during the experiments, introducing a higher runtime variance between the executions.
The personal computer on the other hand was a more controlled environment with no other users and as little processes as possible running at the same time.
Due to this we expect the standard deviation (plotted as error bars in our plots) to be much lower on the PC runs.

We executed every experiment 10 times and calculated the average of all values as well as their standard deviation.
Note that the error bars in all plots represent this standard deviation.

\section{Encrypted Inference with Built-In Models}

These are the results of our first group of experiments, described in \cref{encrypted_inference}.
During reading the results please note that we did not execute the Neural Network experiment (see \cref{neural_net_exp}) on all variations of the SMS Spam dataset (only up to a feature size of 100) because of time constraints.

\subsection{Accuracy and F1-Score}

In \cref{fig:inference_acc_f1} we can see an overview of the accuracy and F1-Scores of our three models.
A full overview of all the data is available in \cref{table:inference_cluster_acc-f1}.

\begin{figure}[h]
    \centering
    \begin{subfigure}[t]{0.49\textwidth}
        \begin{center}
            \includegraphics[width=0.95\textwidth]{../figures/inference_acc_f1-plot_Logistic Regression.pdf}
        \end{center}
        \caption{Logistic Regression}
    \end{subfigure}
    \hfill
    \begin{subfigure}[t]{0.49\textwidth}
        \begin{center}
            \includegraphics[width=0.95\textwidth]{../figures/inference_acc_f1-plot_XGB Tree-based Classification.pdf}
        \end{center}
        \caption{XGB Tree-based}
    \end{subfigure}
    \begin{subfigure}[t]{0.49\textwidth}
        \begin{center}
            \includegraphics[width=0.95\textwidth]{../figures/inference_acc_f1-plot_Neural Network Classifier.pdf}
        \end{center}
        \caption{Neural Network}
    \end{subfigure}
    \caption{Accuracy and F1-Scores of all of our three models in both clear and FHE on various datasets, drawn from the cluster run of the experiments}
\end{figure}\label{fig:inference_acc_f1}

As expected (see \cref{LogisticRegressionExp}) the Logistic Regression model quality did not suffer from compiling it to its \ac{fhe} equivalent.
For all datasets, the accuracy and F1-scores stay exactly the same while also having a no standard deviation (since we fixed the seed for this experiment).
This proofs that with a high quantization bit-width and a very shallow \ac{fhe} inference circuit one can achieve perfect accuracy of the \ac{fhe} model.

Our \ac{cart} model (see \cref{XGBExp}) only suffers from an extremely small accuracy loss due to quantization which is not viewable on the plot.
While this was better than expected, we also have to note that the \ac{pca} that we used in this experiment might have helped a lot here.
In fact it might have even been much too aggressive since in theory any \ac{cart} model must be at least equally as good and most of the time better than Logistic Regression.
What we observed instead were changing improvements over Logistic Regression on some datasets (xor, most spam variants), but a worse performance than Logistic Regression on the others.
As with most experiments we implemented them as close as the example code from the concrete-ml documentation.
In this case however the recommended usage of \ac{pca} probably crippled the model quite a bit to the point were using it over Logistic Regression makes no sense considering its increased runtimes.

For our \ac{fnn} model (see \cref{neural_net_exp}) experiments we did not use a fixed seed since it is not based on scikit-learn anymore and thus concrete-ml does not expose an attribute for doing this.
As a result our accuracies and F1-scores were subject to variation from run to run which can be seen in the error bars.
The \ac{fnn} model has slightly better accuracies across the board, except for the spam and cancer datasets where it is slightly behind Logistic Regression.
The differences are so small however that they are mostly insignificant hinting at the fact that our datasets are not challenging enough to show the differences in model performance well.
Notably the neural network is the only model having acceptable accuracy in our XOR dataset.
As expected the accuracy drops quite much after compiling the model to \ac{fhe}.
How much was highly dependent on the dataset.
While the digits and spam datasets where barely affected, the \ac{fhe} model performed much poorer than the clear model on the cancer and xor datasets which are very challenging for the quantization.

\subsection{Runtime}

\begin{figure}[h]
    \centering
    \begin{subfigure}[t]{0.49\textwidth}
        \begin{center}
            \includegraphics[width=0.95\textwidth]{../figures/inference_runtime-plot_Logistic Regression.pdf}
        \end{center}
        \caption{Both FHE and clear runtimes in absolute numbers on a logarithmic scale}
    \end{subfigure}
    \hfill
    \begin{subfigure}[t]{0.49\textwidth}
        \begin{center}
            \includegraphics[width=0.95\textwidth]{../figures/inference_runtime-plot-with-ratio_Logistic Regression.pdf}
        \end{center}
        \caption{The FHE runtimes relative to the clear runtimes on a linear scale}
    \end{subfigure}
    \caption{Runtimes of the Logistic Regression model on various datasets, drawn from the cluster run of the experiments}\label{fig:inference_runtime_logReg}
\end{figure}

\begin{figure}[h]
    \centering
    \begin{subfigure}[t]{0.49\textwidth}
        \begin{center}
            \includegraphics[width=0.95\textwidth]{../figures/inference_runtime-plot_XGB Tree-based Classification.pdf}
        \end{center}
        \caption{Both FHE and clear runtimes in absolute numbers on a logarithmic scale}
    \end{subfigure}
    \hfill
    \begin{subfigure}[t]{0.49\textwidth}
        \begin{center}
            \includegraphics[width=0.95\textwidth]{../figures/inference_runtime-plot-with-ratio_XGB Tree-based Classification.pdf}
        \end{center}
        \caption{The FHE runtimes relative to the clear runtimes on a linear scale}
    \end{subfigure}
    \caption{Runtimes of the XGB tree-based model on various datasets, drawn from the cluster run of the experiments}\label{fig:inference_runtime_xgb}
\end{figure}

\begin{figure}[h]
    \centering
    \begin{subfigure}[t]{0.49\textwidth}
        \begin{center}
            \includegraphics[width=0.95\textwidth]{../figures/inference_runtime-plot_Neural Network Classifier.pdf}
        \end{center}
        \caption{Both FHE and clear runtimes in absolute numbers on a logarithmic scale}
    \end{subfigure}
    \hfill
    \begin{subfigure}[t]{0.49\textwidth}
        \begin{center}
            \includegraphics[width=0.95\textwidth]{../figures/inference_runtime-plot-with-ratio_Neural Network Classifier.pdf}
        \end{center}
        \caption{The FHE runtimes relative to the clear runtimes on a linear scale}
    \end{subfigure}
    \caption{Runtimes of the Neural Network model on various datasets, drawn from the cluster run of the experiments}\label{fig:inference_runtime_neural_net}
\end{figure}

\subsection{How Runtime Grows with Feature Size}

\subsection{Changes in the Decision Boundary}

\section{Encrypted Training with Built-In Model}

These are the results of our second group of experiments, described in \cref{encrypted_training}.

\subsection{Limitations of concrete-ml and Encountered Errors}

\subsection{Accuracy and F1-Score}

\subsection{Runtime}

\subsection{How Runtime Grows with Feature Size}

\subsection{Changes in the Decision Boundary}

\section{Encrypted Inference with custom approaches for the \acl{ner} task}

These are the results of our third and final group of experiments, described in \cref{ner_exps}.

\subsection{Limitations of concrete-ml and Encountered Errors}

\subsubsection{Imposed Restrictions to Model Design}

\begin{lstlisting}[language=Python, caption=Original model design, label=code:original_model]
class NERModel(nn.Module):
    def __init__(<...>) -> None:
        super().__init__()

        self.window_size = window_size
        self.token_embed = nn.Embedding(vocab_size, embedding_dim)
        capit_embedding_dim = embedding_dim // 2
        self.capit_embed = nn.Embedding(capit_classes, capit_embedding_dim)
        input_dim = (embedding_dim + capit_embedding_dim + 1) * window_size

        layers = []
        prev_dim = input_dim
        for h_dim in hidden_dims:
            layers.append(nn.Linear(prev_dim, h_dim))
            layers.append(nn.ReLU())
            layers.append(nn.Dropout(dropout_rate))
            prev_dim = h_dim
        layers.append(nn.Linear(prev_dim, num_labels))  # Output layer

        self.mlp = nn.Sequential(*layers)

    def forward(self, x):
        token_idxs = x[:, : self.window_size]
        capits = x[:, self.window_size : 2 * self.window_size]
        wlengths = x[:, 2 * self.window_size :]

        token_embeds = self.token_embed(token_idxs)
        capit_embeds = self.capit_embed(capits)
        wlengths = wlengths.float().unsqueeze(
            -1
        )  # since the embeds are 3D, batches of lists of 5 128-dim vectors
        features = torch.cat([token_embeds, capit_embeds, wlengths], dim=-1)
        features_flattened = torch.flatten(features, start_dim=1, end_dim=-1)
        return self.mlp(features_flattened)
\end{lstlisting}

In \cref{code:original_model} you can see our model design as we originally envisioned it.
The goal was to have token indices (token\_idxs) as the main feature.
These where generated by a simple vocabulary, where each token in the training set just got an index (after normalization).
\ac{ner} requires the model to learn semantic information from the surrounding words of the token, i.e. the model needs to understand context.
For this embeddings are a natural choice, specifically PyTorch's Embedding layer.

In addition the these main features we also wanted to add some auxiliary features as well in order to give the model some information that got lost during the normalization and indexation steps.
Some standard features for this are capitalization information and word lengths.
For the capitalization we assigned every token to one of four categories: 0 if the token has no capital letters, 1 if any character in the token is a capital letter, 2 if the first character is a capital letter, and 3 if all characters (except for '-', '\_', '.') are capital letters.
The word lengths are just integers between 1 and 20, where every token longer than 20 characters would also get the length 20 assigned to it.
The capitalization should also go through an embedding of half of the dimensionality than the token embedding, while the word length would get passed to the model directly.

\begin{lstlisting}[caption=Encountered Exception while compiling original model, label=code:original_model_exception]
    File "<path to venv>/lib/python3.11/site-packages/concrete/ml/common/debugging/custom_assert.py", line 26, in _custom_assert
    raise error_type(on_error_msg)
AssertionError: Values must be float if value_is_float is set to True, got int64: <the features>
\end{lstlisting}

However while converting this model to a \ac{fhe} equivalent with the \verb|compile_torch_model| function we encountered the error shown in \cref{code:original_model_exception}.
concrete-ml complained about the provided features being long values, even though that is what is required for an embedding.
Changing the features to floats didn't work since then the PyTorch nn.Embedding would complain about it requiring floats.
Since PyTorch nn.Embeddings were officially supported by concrete-ml, we suspected that the problem was that we split the features first before passing them to the embedding, not triggering a condition that would turn off this assertion for embeddings in the process.
Passing the three different kind of features to the model in three separate variables also did not work and is seemingly not supported by concrete-ml.
To work around this we had to simplify our model to remove the need to split the feature vector (TODO: insert new model).

\begin{lstlisting}[caption=Another Exception encountered during the client preprocessing step, label=code:original_model_exception_2]
    ValueError: Expected argument 0 to be EncryptedTensor<uint12, shape=(1, 5, 20995)> but it's EncryptedTensor<uint6, shape=(1, 5)>
\end{lstlisting}

Another error we encountered related to the PyTorch Embedding is shown in \cref{code:original_model_exception_2}.
This error would only show up in a deployment scenario of the \ac{fhe} model (using the \verb|FHEModelDev|, \verb|FHEModelClient|, and \verb|FHEModelServer| components) that splits up the pre-processing (quantization, encryption, and serialization), processing (the actual inference), and post-processing (deserialization, decryption, and dequantization) into separate steps, and not while running the \verb|forward| method of the model that combines all this steps into one.

\subsection{Accuracy and F1-Score}

\subsection{Runtime}

\include{conclusion_and_outlook}
%%%%%%%%%%%%%%%%%%%%%%%%%%%%%%%%%%%%%%%%%%%%%%%%%%%%%%%%%%%%

% List of figures (Abbildungsverzeichnis):
\listoffigures

% List of tables (Tabellenverzeichnis):
%\listoftables

% References (Literaturverzeichnis):
% a) Style (with abbreviations: use alpha):
% see
% https://de.wikibooks.org/wiki/LaTeX-W%C3%B6rterbuch:_bibliographystyle
% for the different formats and styles

%\bibliographystyle{plainnat}
% b) The File:
%\bibliography{references}

%I switched to BibLatex
\printbibliography

\end{document}
